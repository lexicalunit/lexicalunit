\documentclass[12pt, tweaklist, line]{res}

% TODO:
% 1. Visually uneven?
% 2. Resume doesn't include a summary section -- the key component to compel the
%    hiring manager to keep reading. The career summary content should provide
%    hiring managers with a brief, yet detailed synopsis of what you bring to
%    the table. The purpose of this section is to define you as a professional
%    and cover those areas most relevant to your career level and job target.
% 3. From the wording of the resume, you come across as a "doer," not an
%    "achiever." Too man-y of your job descriptions are task-based and not
%    results-based. This means that they tell what you did, instead of what you
%    achieved. This is a common mistake for non- professional resume writers.
%    To be effective and create excitement, a great resume helps the hiring
%    executive picture you delivering similar achievements at his or her
%    company. Here are some examples of task-based sentences in your resume:
%    - "Worked with services using Ruby, Python, Scala, Kafka, ..."
%    - "Architected aggregation service built on Druid.io, ..."
% 4. Employers want to know about your previous contributions and specifically
%    how you’ve made a difference. More importantly, they want to know how you
%    are going to make a significant difference at their company.
% TOP KEYWORDS & SKILLS: real-time testing python engineer software engineer

\usepackage{hyperref}
\hypersetup{
    colorlinks=true,
    urlcolor=blue
}
% Change global itemize settings
\let\tempone\itemize
\let\temptwo\enditemize
\renewenvironment{itemize}{\tempone\vspace{-.15in}\setlength{\topsep}{0pt}\setlength{\itemsep}{3pt}\vspace{-.15in}}{\temptwo}

% Macros
\def\Cplusplus{{\rm C\raise.5ex\hbox{\small ++}}}
\def\first{{\raise.5ex\hbox{\small st}}}
\def\second{{\raise.5ex\hbox{\small nd}}}
\def\third{{\raise.5ex\hbox{\small rd}}}
\def\fourth{{\raise.5ex\hbox{\small th}}}

% Use more of the page than the default
\addtolength{\oddsidemargin}{-0.50in}
\addtolength{\voffset}{-0.50in}
\addtolength{\textwidth}{0.90in}
\addtolength{\textheight}{1.70in}

\begin{document}

% <Header>

\begin{resume}

% Experience
\section{Experience}
\begin{format}
  \employer{l}\location{r}\\
  \title{l}\dates{r}\\
  \body\\
\end{format}

% UACF
\employer{\textbf{UnderArmour Connected Fitness}}
\title{Senior Software Engineer}
\location{Austin, TX}
\dates{10/2015 -- Present}
\begin{position}
\begin{itemize}
\item Lead developer of Scala based microservice for a \$1\,M Samsung partnership
\item Architected an iOS \& Python \textit{Find Fitness Classes} feature in partnership with Mindbody
\item Used scikit-learn in Python to build classifier for identifying different kinds of fitness classes
\item Refactored existing microservices to use a new real-time stream based notification system, driving 20\,k push notifications and emails per minute
\item Planned and implemented the migration of the MyFitnessPal steps tracking feature used by 225\,M users from a Rails monolith to a Go microservice, reducing response times by 300\%
\end{itemize}
\end{position}

% RMN
\employer{\textbf{RetailMeNot, Inc.}}
\title{Software Engineer II - Data \& Products}
\location{Austin, TX}
\dates{04/2014 -- 09/2015}
\begin{position}
\begin{itemize}
\item Instrumented iOS application with event tracking and built new Products centric UI feature
\item Prototyped a ``Buy Now'' button using \href{https://stripe.com/}{Stripe.js}, a cross-device embeddable payment form
% \item cerebro-deploy fabric based AWS deployment
% \item storm topology
% \item kinesis autoscaling
\item Implemented a realtime high availability data pipeline using Kinesis and Storm
\item Optimized Veg-o-Matic, a SQL based data slicing and reporting tool for A/B testing
\item Architected aggregation service built on \href{http://druid.io/}{Druid.io}, an OLAP realtime datastore
% \item Wrote Python ETL code leveraging our MasterMind framework
% \item Built and maintained other internal data oriented products for Business Intelligence division
\end{itemize}
\end{position}

% Adometry
\employer{\textbf{Adometry, Inc.} (\textbf{Adometry by Google})}
\title{Senior Software Developer - Data Sciences}
\location{Austin, TX}
\dates{03/2012 -- 02/2014}
\begin{position}
\begin{itemize}
\item Leveraged sophisticated data mining techniques to solve big data analytics problems
\item Sharpened \Cplusplus~and Python skills by attending PyCon, PyData, and \Cplusplus~conferences
\item Designed and implemented robust Web API resources using Flask, gunicorn, and Nginx
\item Collaborated with research and sales teams to produce client driven SaaS products
\end{itemize}
\end{position}

% ARL
\employer{\textbf{Applied Research Laboratories: UT}}
\title{Engineering Scientist Associate} % ATL
\location{Austin, TX}
\dates{08/2006 -- 02/2012}
\begin{position}
\begin{itemize}
\item Designed high frequency active sonar installed on 75\% of US nuclear attack submarines
\item Implemented a target tracking and feature classification system for mine avoidance
\item Wrote real-time simulation software for stress testing, debugging, and Navy certification
% \item Devised a VCS to manage security classifications across more than a dozen projects
% \item Developed a high level of expertise with modern \Cplusplus~techniques and design
% \item Traveled to remote contractor sites to integrate ARL software with other US Navy systems
% \item Used Doxygen to generate project and source level documentation for a 3000 KLOC project
% \item Spent significant time working on Gnome, KDE, Motif, and Windows environments
% \item Developed Xlib, Xt, and GTK applications
% \item Built internal tools such as byteswap, bundle, router
% \item Worked with CORBA based DSS system
% \item Was heavily involved in development on the DDG 1000 Zumwalt class Destroyer
% \item Collaborated with NSWC:PC on HFWB
% \item Often worked with Lockheed Martin on ARCI development
\end{itemize}
\end{position}

% SCAL
\employer{\textbf{Special Core Analysis Laboratories, Inc.}}
\title{Software Engineer}
\location{Midland, TX}
\dates{12/2003 -- 01/2006}
\begin{position}
\begin{itemize}
\item Programmed interfaces to data acquisition modules for lab machines
% \item Designed a mobile desorption oven as well as porosity and core gamma systems
\end{itemize}
\end{position}

\pagebreak

% <Header>
\vspace{-.17in}
\vtop{\hspace{-.55in} \hsize=6.9in \hrulefill}

% Open Source
\vspace{-.08in}
\section{Open Source}
\begin{itemize}
\vspace{.55in} % why do I need to manually put vertical space here?
% \item OSS contributions in the form of PRs and such to Atom packages and bokeh and JSONSchemaValidation...
\item \href{http://nanodbc.io}{nanodbc}: A small \Cplusplus~ wrapper for the native C ODBC API
\item \href{http://pancake.lexicalunit.com}{pancake-master}: A dynamic webpage for Master Pancake showtimes
\item \href{https://atom.io/packages/event-watch}{event-watch}: Recurring event watch plugin for the Atom editor
\item \href{https://atom.io/packages/multi-wrap-guide}{multi-wrap-guide}: Multi-wrap guides for the Atom editor, \href{http://blog.atom.io/2015/08/06/new-package-roundup.html}{featured in an Atom blog post!}
\item \href{https://github.com/lexicalunit/atom-notes}{atom-notes}: Embedded notational velocity for Atom
\item \href{https://github.com/lexicalunit/dict}{\Cplusplus~dict}: Proof-of-concept \Cplusplus~dict class with Python-like features
\item \href{https://github.com/lexicalunit/dotfiles}{dotfiles}: My personalized machine configuration management system
\end{itemize}

% Education
\section{Education}
\begin{format}
  \employer{l}\location{r}\\
  \title{l}\dates{r}\\
  \body\\
\end{format}

% TAMU
\employer{\textbf{Texas A\&M University}}
\title{Bachelor of Science in Computer Science}
\location{College Station, TX}
\dates{08/2001 -- 05/2006}
\begin{position}
% GPA: 3.4\\
\begin{itemize}
% \item Minors in Linguistics and Mathematics
\item Implemented a buddy-system memory manager, and multi-level feedback process scheduler
\item Coded in assembly for MIPS processor and modeled a pipelined processor in Verilog
\end{itemize}
\end{position}

% Skills
\section{Skills}

~\\ % why is this necessary?

\begin{itemize}
\item \textbf{Languages:} Python, Scala, ruby, Go, JavaScript/ES6, Zsh/Bash, C/Obj-C/\Cplusplus, Java
\item \textbf{Tools:} git, Docker, Node.js, Kubernetes, JIRA/Confluence, Jenkins, CMake
\item \textbf{Services:} Amazon Web Services (AWS), Kibana, Librato, Kafka, Sentry, PagerDuty
\item \textbf{Databases:} MySQL, DynamoDB, Storm, Druid.io, MongoDB, Vertica/Greenplum
\item \textbf{Libraries:} AKKA, Finatra, Flask, Ruby on Rails, Boost, fabric, SQLAlchemy
\item \textbf{Practices:} Agile, SCRUM, Test Driven \& Behavior Driven Development
\item \textbf{Markup:} YAML, JSON, HTML/CSS/LESS/Sass, Markdown, \LaTeX, ERB, Wiki
\item \textbf{Operating Systems:} MacOS, Linux (RedHat and Debian based), Windows
\end{itemize}

% Portfolio
% \section{Portfolio}
% \begin{itemize}
% \vspace{.55in} % why do I need to manually put vertical space here?
% \item My personal webpage: \href{http://lexicalunit.com}{lexicalunit.com}
% \item My open source ODBC wrapper: \href{http://nanodbc.io}{nanodbc.io}
% \item My open source projects: \href{https://github.com/lexicalunit}{github.com/lexicalunit}
% \item My contributions to the Atom Editor: \href{https://atom.io/users/lexicalunit}{atom.io/users/lexicalunit}
% \item My LinkedIn user profile: \href{https://www.linkedin.com/in/amy-troschinetz-53426b85/}{linkedin.com/in/amy-troschinetz-53426b85}
% \item My UACF \textit{Makers Blog} posts: \href{https://makers.underarmour.com/author/amytroschinetz/}{makers.underarmour.com/author/amytroschinetz}
% \end{itemize}

\end{resume}
\end{document}
